\chapter{实验}
\label{cha:experiment}
在本章中,我们分别介绍对某一目标的情感分析(Aspect-Term Sentiment Analysis or Target-oriented Sentiment Analysis)和新闻驱动的股票预测的实验设置及实验结果。

\section{对某一目标的情感分析}
在本小节,我们介绍对某一目标的情感分析(Aspect-Term Sentiment Analysis or Target-oriented Sentiment Analysis)的实验。这里,我们使用的是基于transformer和多尺度卷积的模型。
\subsection{实验设置}
我们在Restaurant、Laptop、Twitter三个公开数据集上进行了实验。这三个数据集均在第~\ref{cha:data}中有所介绍。我们沿用了之前工作中数据预处理的方法~\cite{Xin2018Transformation,Tang2015Effective},我们除去了一些冲突的标签,所有单词均转化为小写,不去除任何的停止词、符号、数字。我们使用nltk\footnote{http://www.nltk.org/}工具提供的分词工具对全部的句子进行分词。所有的句子都使用“PAD”字符补齐成最大长度。

我们使用准确率和macro-averaged F1 score作为评价指标。对每一类来说,精确率$P=\frac{TP}{TP+FN}$,召回率$R=\frac{TP}{TP+FP}$,F1 score由$\frac{2PR}{P+R}$求得。Macro-averaged F1 score是所有类别的F1 score的均值~\cite{Tang2015Effective}。

我们与下列模型进行了对比:
\begin{itemize}
    \item Majority把训练集中出现频率最高的类别作为预测类别;
    \item SVM是支持向量机模型,它使用的是人为构建的n-gram特征、解析特征和语义特征~\cite{kiritchenko2014nrc-canada-2014:};
    \item AE-LSTM是一个将目标和句子拼接起来作为输入的LSTM模型;
    \item ATAE-LSTM对AE-LSTM进行了扩展,利用attention机制选取最重要的单词~\cite{wang2016attention-based};
    \item IAN利用交互式地方式,分别使用attention方法计算句子和目标的表示~\cite{ma2017interactive};
    \item BILSTM-ATT-G使用控制门来控制目标左边、右边部分的重要性~\cite{liu2017attention};
    \item GCAE使用卷积神经网络和控制门,得到了更高的准确率,更容易并行化~\cite{xue2018aspect};
    \item {Memnet}将词向量当作记忆单元,使用多层attention方法来得到最终表示,为克服attention机制不能获取时序信息的缺点,它还使用了位置权重~\cite{tang2016aspect};
    \item RAM对Memnet做出了改进,它使用双向LSTM的结果作为记忆单元,使用GRU来生成下一层的表示,同时使用了与Memnet不同的位置权重~\cite{Al2017Deep};
    \item TNet提出生成与目标相关的句子表示,同时结合上下文信息。\cite{Xin2018Transformation}.
\end{itemize}
我们使用pytorch\footnote{https://pytorch.org}框架实现了这些模型中除IAN之外的模型,并使它们的实验结果尽可能地与原论文相似。每个模型都是独立训练的。我们使用的是GloVe~\cite{pennington2014glove}词向量来初始化我们的词向量,并在训练的过程中调整词向量的值。
在我们的模型中,我们使用的模型参数设置与OpenAI GPT~\cite{radford2018improving}一致。具体地,transformer的层数是12,多头attention的head数是12,词向量的维度设为768,中间层的维度设为了3072。我们首先加载OpenAI GPT~\cite{radford2018improving}的预训练参数,然后与后续结构一起进行调优。我们使用了五个不同的卷积核,卷积核大小从1到5。卷积的输出通道数设为100。我们使用Adam~\cite{kingma2014adam}优化器,学习率设为6.25e-5。模型在20轮训练内就得到了最好的效果。
\subsection{实验结果}

\begin{table}[ht]
    \centering
    \caption{对某一目标的情感分析实验结果(\%)。带“*”的实验结果是从原文中获取的。}
    \label{tab:result}
    \begin{tabular}{lcccccc}
    \hlinewd{1.2pt}
    \multirow{2}{*}{Models} & \multicolumn{2}{l}{Restaurant} & \multicolumn{2}{l}{Laptop} & \multicolumn{2}{l}{Twitter} \\ \cline{2-7} 
                            & ACC       & Macro-F1      & ACC     & Macro-F1    & ACC     & Macro-F1     \\ \hlinewd{1.2pt}
    Majority                &65.00           &-              &53.45         &-            &50.00         &22.22         \\
    SVM                     &80.89           &-              &72.10         &-            &63.40         &63.30         \\ 
    LSTM                    &76.70           &63.57          &69.28         &63.30        &66.04         &63.46         \\
    ATAE-LSTM               &77.23           &63.73          &69.44         &63.46        &71.24         &69.19         \\
    IAN                     &78.60$^*$       &-              &72.10$^*$     &-            &-             &-             \\
    BILSTM-ATT-G            &79.20           &67.07          &71.32         &64.88        &71.68         &70.37         \\
    GCAE                    &78.12           &62.50          &70.38         &64.02        &72.40         &70.89         \\
    MemNet                  &77.86           &64.47          &68.18         &62.46        &69.80         &66.86         \\
    RAM                     &78.30           &65.42          &71.63         &66.73        &71.24         &68.75         \\
    TNet                    &78.39           &65.37          &73.98         &68.64        &72.11         &70.01         \\ \hline
    Ours                    &84.20           &76.35          &78.21         &73.31        &72.98         &71.40         \\ \hlinewd{1.2pt}
    \end{tabular}
    \end{table}
表~\ref{tab:result}显示了对某一目标的情感分析的实验结果。我们的模型在Restaurant、Laptop、Twitter数据集上均取得了最佳的实验结果。我们在Restaurant数据集上取得了84.20\%的准确率(提升5.81\%),在Laptop数据集上取得了78.21\%的准确率(提升4.23\%),在推特数据集上取得了72.98\%的准确率(提升0.87\%)。

在所有的神经网络模型中,LSTM模型效果最差。ATAE-LSTM考虑到了目标并使用了attention方法,效果有所提升。IAN使用了句子对目标的attention和目标对句子的attention,更进一步提升了实验效果。BILSTM-ATT-G和RAM在Restaurant和Laptop上的实验效果较好,但在Twitter数据集上效果提升并没有那么大,可见LSTM不擅长处理Twitter数据中大量的口语化文本。TNet使用了CNN和LSTM,并在三个数据集上都取得了不错的效果。

不同于这些模型,我们的模型使用transformer提取句子特征,更易于处理长期依赖关系,易于并行。我们的模型利用多尺度卷积,从不同粒度提取特征。我们的模型在三个数据集上均取得了最佳的效果。
\subsection{预训练}
为了证明transformer预训练的效果,我们进行了对比实验:一组使用预训练的transformer参数,另一组完全随机初始化参数,并从头开始进行训练。
\begin{table}[h]
    \centering
    \caption{预训练的作用}
    \label{tab:pretrain}
    \begin{tabular}{lcccc}
         \hlinewd{1.2pt}
     \multirow{2}{*}{\textbf{Models}} & \multicolumn{2}{c}{Restaurant} & \multicolumn{2}{c}{Laptop} \\
     \cline{2-5}
                                      & ACC & Marco-F1 & ACC & Marco-F1 \\
    \hlinewd{1.2pt}
         w/o pre-training & $69.20$ & $48.16$ & $64.89$ & $59.25$ \\
         \hline
         w/ pre-training & $84.20$ & $76.35$ & $78.21$ & $73.31$ \\
    \hlinewd{1.2pt}
    \end{tabular}
\end{table}
表~\ref{tab:pretrain}显示了对比实验的实验结果。当不使用预训练的参数时,实验效果大大下降。在大规模语料下的预训练,使得transformer可以获取句子的语义信息。

\subsection{多尺度卷积}
为证明多尺度卷积的重要性,我们将模型中的多尺度卷积去掉,对比了简化后的模型与完整模型的实验效果。
\begin{table}[h]
    \centering
    \caption{多尺度卷积的作用}
    \label{tab:cnn}
    \begin{tabular}{lcccc}
         \hlinewd{1.2pt}
     \multirow{2}{*}{\textbf{Models}} & \multicolumn{2}{c}{Restaurant} & \multicolumn{2}{c}{Laptop} \\
     \cline{2-5}
                                      & ACC & Marco-F1 & ACC & Marco-F1 \\
    \hlinewd{1.2pt}
         w/o cnn & $83.39$ & $74.40$ & $77.43$ & $72.42$ \\
         \hline
         w/ cnn & $84.20$ & $76.35$ & $78.21$ & $73.31$ \\
    \hlinewd{1.2pt}
    \end{tabular}
\end{table}
表~\ref{tab:cnn}显示了对比实验的结果。多尺度卷积能够利用全部单词的表示信息,提取不同长度的短语的表示,并选取最为重要的信息。多尺度卷积将Restaurant数据集上的准确率提升约0.81\%,将Laptop数据集上的准确率提升约0.7\%。
\subsection{样例分析}
\begin{table}[ht]
    \centering
    \caption{结果示例。输入目标用中括号标记,正确输出标签以标的形式给出}
    \label{tab:case}
    \begin{tabular}{m{7cm}|>{\centering\arraybackslash}m{1cm}>{\centering\arraybackslash}m{1cm}>{\centering\arraybackslash}m{1cm}}
    \hlinewd{1.2pt}
    Sentence & ATAE-LSTM & GCAE & Ours  \\ \hline
    {[}Coffee{]}$_P$ is a better deal than overpriced sandwiches.      
    & $P$  & $O$\textsuperscript{\xmark} & $P$ \\ \hline
    But make sure you have enough room on your credit card as the {[}bill{]}$_P$ will leave a big dent in your wallet. 
    & $P$ & $O$\textsuperscript{\xmark} & $P$ \\ \hline
    Aww, it 's okay... You have a {[}PSP{]}$_P$. :D That 's good already.
    & $O$\textsuperscript{\xmark} & $P$ & $P$ \\ \hline
    I hate my {[}iPod{]}$_N$! It's dead! dead dead dead! ! ! Someone wanna fix it for me?
    & $O$\textsuperscript{\xmark} & $N$ & $N$ \\ \hline
    I have never had a bad {[}meal{]}$_P$ (or bad service) at pigalle.
    & $N$\textsuperscript{\xmark} & $N$\textsuperscript{\xmark} & $P$ \\ \hline
    The {[}staff{]}$_N$ should be a bit more friendly. 
    & $P$\textsuperscript{\xmark} & $P$\textsuperscript{\xmark} & $N$ \\ \hline
    It's a basic pizza joint, not much to look at, but the {[}pizza{]}$_P$ is what I go for.
    & $N$\textsuperscript{\xmark} & $O$\textsuperscript{\xmark} & $P$ \\ \hline
    \hlinewd{1.2pt}
    \end{tabular}
\end{table}
表~\ref{tab:case}显示了一些预测的例子。输入目标用中括号标记,正确输出标签以下标的形式给出。其中$P$、$N$、$O$分别表示积极、消极和中立。比如,在第一个句子中,对“coffee”的情感极性是积极的。我们的模型比ATAE-LSTM和GCAE的预测结果要好。前两行中,句子的句式比较正式,LSTM的模型更擅长解决此类问题。后两行中,句子比较口语化,句子结构比较零散,CNN更擅长解决此类问题。Transformer的模型经过了大规模语料库的预训练,在处理这两种情况时都有不错的表现。另外,我们的模型具备一定的推理和比较的能力,第五、六行显示了相应的例子。最后一行,表示情感的词是“what I go for”,因为使用了多尺度卷积,我们的模型可以提取到短语的信息。
\section{新闻驱动的股票预测}
在本小节中,我们介绍新闻驱动的股票预测相关实验。我们使用的是MSRA模型(Multi-Stock Relation model using Attention mechanism,使用attention机制的多股票关系模型)。
\subsection{实验设置}
我们在前面提到的senti-stock数据集上进行实验。所有单词都转化为小写,不移除任何停止词、符号或数字。我们使用nltk\footnote{http://www.nltk.org/}工具提供的分词工具对全部的句子进行分词。所有句子用"PAD"符号补齐到最大长度。我们以准确率和F1平均值为主要的评价指标。对每一类而言,精确率$P=\frac{TP}{TP+FN}$,召回率$R=\frac{TP}{TP+FP}$,F1值由$\frac{2PR}{P+R}$求得。其中,$TP$、$TN$、$FN$、$FP$分别表示真正例、真负例、假负例、假正例的个数。最终的F1平均值是各个类的F1值的均值。相比于准确率,F1值更能评价在不均衡数据集下预测结果的好坏。

在前面提到了基于方面的情感分析模型中,有一些模型要求目标必须出现在句子当中,或目标不止包含有一个单词,因此不能应用到新闻驱动的股票预测这一场景当中。因此,我们的模型只与下列模型进行了对比:
\begin{itemize}
    \item Majority将训练集中大多数的标签作为测试集的预测标签。
    \item LSTM利用简单的LSTM来处理新闻句子信息,不考虑股票的相关信息。
    \item ATAE-LSTM在LSTM的基础上,引入了attention机制,结合了股票信息和新闻句子信息。
    \item GCAE使用CNN来处理新闻句子,通过CNN计算得到了gate来调整预测的结果。
\end{itemize}
我们使用pytorch\footnote{https://pytorch.org}复现了上述模型。每个模型都是独立训练的。每个模型都使用GloVe~\cite{pennington2014glove:}预训练的词向来初始化词向量,词向量的维度是300。CNN的卷积核大小从3到5,卷积的输出通道数设为100。我们使用AdaGrad~\cite{duchi2011adaptive}优化器,学习率设为0.01。模型的实现是开源的\footnote{https://www.github.com/Cppowboy/icann2019}。
\subsection{实验结果}
\begin{table}[ht]
    \centering
    \caption{senti-stock数据集上的实验结果}
    \label{tab:mainresult}
    \begin{tabular}{lcccccc}
    \hline
    \multirow{2}{*}{Models} & \multicolumn{2}{c}{Short}         & \multicolumn{2}{c}{Middle}        & \multicolumn{2}{c}{Long}                                \\ \cline{2-7} 
                            & Accuracy        & Macro-F1        & Accuracy        & Macro-F1        & Accuracy                   & Macro-F1                   \\ \hline
    Majority                & 0.5710          & 0.2423          & 0.3832          & 0.1847          & 0.4789                     & 0.2158                     \\
    LSTM                    & 0.6316          & 0.5542          & 0.5655          & 0.5594          & 0.6162                     & 0.5641                     \\
    ATAE-LSTM               & 0.6401          & 0.5688          & 0.5687          & 0.5663          & 0.6241                     & 0.5791                     \\
    GCAE                    & 0.6828          & 0.6072          & 0.6097          & 0.6047          & 0.6630                     & 0.6161                     \\
    MSRA                    & \textbf{0.6920} & \textbf{0.6198} & \textbf{0.6273} & \textbf{0.6237} & \textbf{0.6826}            & \textbf{0.6387}            \\ \hline
    \end{tabular}
\end{table}
表~\ref{tab:mainresult}显示了在senti-stock数据集上的主要实验结果。LSTM在所有神经网络模型中效果最差,因为它只考虑了新闻句子的信息,没有考虑股票的相关信息。ATAE-LSTM引入了attention机制,考虑了股票相关信息,使得它相比LSTM有所提升。GCAE使用了CNN来处理新闻标题这种比较简洁的文本,取得了更好的效果。我们所提出的模型在三种不同的时间间隔上的预测效果最好。
\begin{figure}[H]
    \centering 
    \includegraphics[width=\linewidth]{short-acc.png}
    \caption{各股票短期预测准确率}
    \label{fig:short-acc}
\end{figure}
\begin{figure}[H]
    \centering 
    \includegraphics[width=\linewidth]{middle-acc.png}
    \caption{各股票中期预测准确率}
    \label{fig:middle-acc}
\end{figure}
\begin{figure}[H]
    \centering 
    \includegraphics[width=\linewidth]{long-acc.png}
    \caption{各股票长期预测准确率}
    \label{fig:long-acc}
\end{figure}
图\ref{fig:short-acc}、图\ref{fig:middle-acc}、图\ref{fig:long-acc}分别显示了各支股票的短期、中期、长期的预测准确率。短期预测和长期预测的准确率明显比中期预测要高,大部分股票的短期和长期预测准确率均超过了百分之五十。而中期预测的准确率相对较低,大部分股票的中期预测准确率超过了百分之四十。谷歌(GOOG)在短期、中期和长期的预测准确率均超过了百分之八十,说明谷歌的股票价格与其相关新闻关系非常密切。
\subsection{模型简化实验}

\begin{table}[ht]
    \centering
    \caption{模型简化实验结果}
    \label{tab:ablation}
    \begin{tabular}{lcccccc}
    \hline
    \multirow{2}{*}{Models} & \multicolumn{2}{c}{Short}         & \multicolumn{2}{c}{Middle}        & \multicolumn{2}{c}{Long}                                \\ \cline{2-7} 
                            & Accuracy        & Macro-F1        & Accuracy        & Macro-F1        & Accuracy                   & Macro-F1                   \\ \hline
    CNN                     & 0.6498          & 0.5786          & 0.5813          & 0.5761          & \multicolumn{1}{c}{0.6490} & \multicolumn{1}{c}{0.5961} \\
    MCNN                    & 0.6616          & 0.5852          & 0.5920          & 0.5871          & \multicolumn{1}{c}{0.6531} & \multicolumn{1}{c}{0.6005} \\
    MCSC                    & \textbf{0.6955} & \textbf{0.6208} & 0.6236          & 0.6205          & 0.6765                            & 0.6275                     \\ 
    MSRA                    & 0.6920          & 0.6198          & \textbf{0.6273} & \textbf{0.6237} & \textbf{0.6826}                   & \textbf{0.6387}            \\ \hline
    \end{tabular}
\end{table}
为了证明MSRA模型各个部分的有效性,我们进行了模型简化实验。CNN是一个基本的卷积神经网络模型,它只考虑新闻句子的输入,不考虑股票的信息。MCNN使用了多尺度卷积,可以提取不同长度的短语的信息。MCSC使用了股票系数,股票系数引入了股票相关信息。MSRA是我们的完整模型,它使用了attention机制,学习股票之间的相关关系。从表~\ref{tab:ablation}可以看到,多尺度卷积和股票系数都可以明显地提升模型的性能。股票关系模型对中期和长期的预测有更大的作用。一种可能的解释是,股票关系模型反映了股票之间的相互关系,短期来看,这些关系可能作用不大,但股票之间的关系对股票长期的变动影响更大。

\subsection{样例分析}

\begin{table}
    \centering  
    \caption{新闻驱动的股票预测样例分析}
    \label{tab:stockcase}
    \begin{tabular}{m{7cm}|l|m{1cm}|m{1cm}|m{1cm}}
        \hline  
        news title & stock & ATAE-LSTM & GCAE & MSRA \\ \hline 
        Tesla delivers quarterly record of 25000 vehicles in first quarter & TSLA\textsubscript{+1} & +1\textsuperscript{\cmark} & -1\textsuperscript{\xmark} & +1\textsuperscript{\cmark} \\  \hline
        Taiwan stocks hit over 18-mth highs TSMC up ahead of Q4 result & TSM\textsubscript{+1} & +1\textsuperscript{\cmark} & 0\textsuperscript{\xmark} & +1\textsuperscript{\cmark} \\ \hline 
        %        Morgan Stanley's profit doubles on bond-trading surge & MS\textsubscript{+1} & -1\textsuperscript{\xmark} & +1\textsuperscript{\cmark} & +1\textsuperscript{\cmark} \\ \hline
        AT\&T to offer hulu to customers this year. & T\textsubscript{+1} & 0\textsuperscript{\xmark} & +1\textsuperscript{\cmark} & +1\textsuperscript{\cmark} \\ \hline
        Consumer reports says Tesla misunderstands 'positive' Model 3 rating & TSLA\textsubscript{-1} & +1\textsuperscript{\xmark} & -1\textsuperscript{\cmark} & -1\textsuperscript{\cmark} \\ \hline
        % Amazon-Apple TV deal shows tough road to cooperation for tech rivals & AMZN\textsubscript{-1} & 0\textsuperscript{\xmark} & -1\textsuperscript{\cmark} & -1\textsuperscript{\cmark} \\ \hline 
        % Apple Google reach new deal with workers xin U.S. lawsuit over hiring & AAPL\textsubscript{+1} & 0\textsuperscript{\xmark} & +1\textsuperscript{\cmark} & +1\textsuperscript{\cmark} \\ \hline
        %        Amazon offers prime discount for U.S. customers on government aid & AMZN\textsubscript{-1} & 0\textsuperscript{\xmark} & 0\textsuperscript{\xmark} & -1\textsuperscript{\cmark} \\ \hline
        % Wal-Mart takes on amazon 's 'prime day ' with online sale & WMT\textsubscript{1} & +1\textsuperscript{\xmark} & +1\textsuperscript{\xmark} & +1\textsuperscript{\xmark} \\ \hline  
        Tesla has recalled 53 000 of its model s model x cars & TSLA\textsubscript{+1} & -1\textsuperscript{\xmark} & -1\textsuperscript{\xmark} & -1\textsuperscript{\xmark} \\ \hline  
    \end{tabular}
\end{table}
为了进一步分析MSRA模型的特点,我们在表\ref{tab:stockcase}中展示了新闻驱动的股票预测的样例,正确预测值以下标的形式给出。我们比较了以ATAE-LSTM为代表的LSTM模型和以GCAE为代表了CNN模型,以及我们所提出的股票关系模型。在前两行所展示的例子中,句子都是比较正式,LSTM更擅长处理此类问题。在第三行的例子中,因为标题的简洁性,标题与普通句子的语法并不完全一致,CNN能够很好地处理此类问题。第四行中,“misunderstands 'positive' rating”,LSTM误将“positive”识别为积极的消息,CNN模型可以将误解与积极联系成一个短语。另外,在最后一行的例子中,模型预测消息是消极的,从人的主观观察来看,这一结果是正确的,但是,它与股票实际的涨跌不同。因为股票的涨跌受到许多因素的影响,新闻并不是决定股票涨跌的唯一因素。为了进一步提升股票预测的准确率,我们可以考虑补充股票历史价格、技术指标等其他信息,这也是后续研究的方向之一。
% \subsection{市场模拟}
% 为了进一步验证我们的模型效果,我们进行了市场模拟实验。我们采用了这样的交易策略\cite{lavrenko2000mining}:我们每天进行一次预测,如果预测某支股票的价格在下一个时间段(天、周、月)内会上涨,则买入价值一万美金的这支股票,如果在下一个时间段内(天、周、月),股票价格上涨超过百分之一,则将其卖出,否则,在下一天、周或月后,将股票卖出;如果预测某支股票价格在下一个时间段(天、周、月)内会下跌,则买空一万美金的这支股票,如果在下一个时间段内(天、周、月),股票价格下跌超过百分之一,则将其卖空,否则,在下一天、周或月后,将股票卖空。我们将初始资金设为三万美金,并利用历史数据进行训练和模拟。我们将购买标普500指数并持有作为基准策略,无风险收益定为百分之三。

% \subsection{策略风险评价指标}

% 策略的风险指标能够从各个维度对策略有客观、全面的评估。本文中,我们使用了以下风险评价指标。
% \subsubsection{年化收益率}
% 年化收益率(Annualized Returns)表示投资一年的预期收益率。
% \begin{equation}
%     p_r=(\frac{p_{end}}{p_{start}})^{\frac{250}{n}}-1
% \end{equation}
% 其中,$p_{end}$是策略最终总资产,$p_{start}$是策略初始总资产,$n$是交易日数量。
% \subsubsection{基准年化收益率}
% 基准年化收益率(Benchmark Returns)表示参考标准年化收益率。
% \begin{equation}
%     B_r=(\frac{B_{end}}{B_{start}})^{\frac{250}{n}}-1
% \end{equation}
% 其中,$B_{end}$是基准最终值,$B_{start}$是基准初始值,$n$是交易日数量。
% \subsubsection{贝塔}
% 贝塔(Beta)表示投资的系统性风险,反映了策略对大盘变化的敏感性。例如,一个策略的Beta为1.3,则大盘涨1\%的时候,策略可能涨1.3\%,反之亦然;如果一个策略的Beta为-1.3,说明大盘涨1\%的时候,策略可能跌1.3\%,反之亦然。
% \begin{equation}
%     \beta=\frac{Cov(p_n,B_n)}{\sigma^2_B}
% \end{equation}
% 其中,$p_n$是策略每日收益率,$B_n$是基准每日收益率,$\sigma^2_B$基准每日收益方差,$Cov(p_n,B_n)$是策略和基准每日收益的协方差。
% \subsubsection{阿尔法}
% 投资中面临着系统性风险(即Beta)和非系统性风险(即Alpha)。Alpha是投资者获得与市场波动无关的回报,一般用来度量投资者的投资技艺。例如,投资者获得了12\%的回报,其基准获得了10\%的回报,那么Alpha或者价值增值的部分就是2\%。
% \begin{equation}
%     \alpha=p_r-r_f-\beta(B_r-r_f)
% \end{equation}
% 其中,$p_r$是策略年化收益率,$r_f$是无风险收益率,$B_r$是基准年化收益率。当Alpha大于零时,说明策略相对于风险,获得了超额收益;当Alpha等于零时,说明策略相对于风险,获得了适当收益;当Alpha小于零时,说明策略相对于风险,获得了较少收益。
% \subsubsection{收益波动率}
% 收益波动率(Volatility)用来测量资产的风险性,波动越大代表策略风险越高。
% \begin{equation}
%     \sigma_p=\sqrt{\frac{250}{n}\sum_{t=1}^{n}(p_t-\overline{p_t})^2}
% \end{equation}
% 其中,$n$是交易日数量,$p_t$是策略每日收益率,$\overline{p_t}=\frac{1}{n}\sum_{t=1}^{n}p_t$策略每日平均收益率。
% \subsubsection{夏普比率}
% 夏普比率(Sharpe Ratio)表示每承受一单位总风险,会产生多少的超额报酬,可以同时对策略的收益与风险进行综合考虑。
% \begin{equation}
%     SharpRatio=\frac{p_r-r_f}{\sigma_p}
% \end{equation}
% 其中,$p_r$是策略年化收益率,$r_f$是无风险收益率,$\sigma_p$是策略收益率波动率
% \subsubsection{信息比率}
% 信息比率(Information Ratio)用来衡量单位超额风险带来的超额收益。信息比率越大,说明该策略单位跟踪误差所获得的超额收益越高,因此,信息比率较大的策略的表现要优于信息比率较小的策略。合理的投资目标应该是在承担适度风险下,尽可能追求高信息比率。
% \begin{equation}
%     InformationRatio=\frac{p_r-B_r}{\sigma_t}
% \end{equation}
% 其中,$p_r$是策略年化收益率,$B_r$是基准年化收益率,$\sigma_t$是策略与基准每日收益差值的年化标准差。
% \subsubsection{最大回辙}
% 最大回撤(Max Drawdown)描述策略可能出现的最糟糕的情况。
% \begin{equation}
%     MaxDrawDown_t=max(1-\frac{P_j}{P_i})
% \end{equation}
% $MaxDrawDown_t$为$t$日的最大回撤,$P_i$和$P_j$分别为$i$日和$j$日的策略总资产,其中$t\ge j>i$。

\subsection{风险提示}
利用MSRA模型进行新闻驱动的股票预测是对历史经验的总结,存在失效的可能。股市有风险,投资须谨慎。