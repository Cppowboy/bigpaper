\chapter{介绍}
\label{cha:introduction}

\section{研究内容}

股票(stock)是股份公司发行的所有权凭证,是股份公司为筹集资金而发行给各个股东作为持股凭证并借以取得股息和红利的一种有价证券。每股股票都代表股东对企业拥有一个基本单位的所有权。每家上市公司都会发行股票。股票趋势预测一直是股票投资所关注的热点之一,优秀的股票预测可以带来巨大的收益。因此,股票预测一直是学术界和金融界共同的研究兴趣。

大量经验性的研究表明,股票在某种程度上是可以被预测的。行业研究员通过基本面研究,由股票公司的运行情况、相关政策、新闻等,股票的估值进行预测;量化分析师通过技术分析,运用数学、统计的方法,从股票历史交易数据中,寻找规律,进行股票趋势预测。而在本文中,我们专注于解决新闻驱动的股票预测问题。新闻事件能够影响交易员的决定,交易员的交易影响股票的价格。因此,我们可以认为,新闻事件是可以影响股票价格的。

随着自然语言处理技术的发展,使用人工智能的方法处理新闻文本,提取相关信息成为了可能。我们提出为每个股票创建一个相应的向量,从而将新闻驱动的股票预测转化为了基于方面的情感分析(其目标是提取句子对某一方面或目标的情感极性)。我们假设股票的涨跌仅受前一个时间段的新闻的影响,通过预测新闻(文本)对股票(目标)的影响(情感极性),来预测股票的涨跌。
\section{研究背景}

近期的研究工作开始应用文本挖掘技术分析新闻对股票市场的影响。研究者使用OpinionFinder~\cite{Wilson2005OpinionFinder}和Google-Profile of Mood States (GPOMS)工具,来研究推特(Twitter)情绪是否会影响道琼指数(Dow Jones Industrial Average, DJIA)\cite{bollen2011twitter}。他们利用上述两款工具处理推特数据,获取了情绪的时间序列,并以此预测道琼指数的每日涨跌,获得了87.6\%的准确率。这一研究表明,网络内容的情感信息与股票市场的变动是相关的。之后,有研究者提出了结构化的事件表示方法\cite{ding2014using}。他们提出使用信息抽取的方法来得到事件的表示,在不使用人力的情况下,从大规模的公开新闻中得到事件的结构化表示,即一个包括主语、谓语、动作和时间的四元组。他们利用线性和非线性的模型,探索新闻事件和股票市场之间的隐含的复杂关系。他们预测标普500的变化的准确率达到了60\%,预测单支股票的准确率超过了70\%。为进一步提升事件驱动的股票预测的准确率,优化事件的表示形式,又有研究者用深度学习的方式来解决这一问题\cite{ding2015deep}。首先,他们从新闻文本中提取事件,用一个向量表示事件,并用一个神经网络对其进行训练;然后,用一个深度卷积网络来预测股票的短期和长期变化。相比于之前的方法,他们将预测标普500和单支股票涨跌的准确率提升了将近6\%。

上述研究工作主要存在两个问题:首先,他们利用词袋模型、结构化四元组、稠密向量的方式来表示事件,随着自然语言处理技术的发展,这些表示方式已经不是新闻事件的最优化表示方式了;然后,这些工作中所用到的数据集大都是不公开的,影响了进一步的后续研究。

\section{本文贡献}

为解决第一个问题,我们利用自然语言处理的技术,直接将新闻文本作为事件输入。同时,我们提出为每一个股票创建一个向量表示,即股票向量的概念。据我们所知,这是第一次提出股票向量的概念。这样,我们将新闻驱动的股票预测转化为预测新闻事件对某支股票的影响。这一问题在自然语言处理中,被称为基于方面的情感分析,其目标是预测一段文本(新闻文本)对某一特定方面(股票)或目标的影响(涨跌)。比如,“微软收购诺基亚手机部门”这一新闻对微软来说是积极的,将会导致微软的股票在下一个时间段内上涨;对诺基亚来说是消极的,将导致诺基亚的股票在下一时间段内下跌。

为解决第二个问题,我们创建了一个新闻驱动的股票预测的数据集。我们从互联网的公开数据中抓取原始数据。利用来自于路透社的新闻数据,以及来自于雅虎财经的股票历史数据,我们构建了名为senti-stock的数据集。在这里,我们假设股票价格的涨跌仅与股票在前一个时间段(前一天、周、月)的相关新闻有关,忽略其他的影响因素。我们将新闻文本以及与这条新闻相关的股票作为输入,将股票在下一个时间段(日、周、月)内的变动情况作为输出标签。具体地,如果股票在下一个时间段内(日、周、月)的涨幅超过1\%,则将输出标签设为$+1$,如果股票在下一个时间段(日、周、月)的跌幅超过1\%,则将输出标签设为$-1$,其他的输出标签设为$0$。

为解决基于方面的情感分析问题,并将其应用于股票预测,我们提出了自己的模型。

首先,为解决LSTM的无法并行化的缺陷和CNN不擅长解决长期依赖的问题,我们提出使用transformer~\cite{NIPS2017_7181}和多尺度卷积的模型来解决基于方面的情感分析问题。Transformer的结构由谷歌提出,它完全抛弃了LSTM和CNN,仅使用self-attention就在自然语言处理的诸多应用中取得了优秀的结果,可以处理长期依赖问题,并且便于并行化。多尺度卷积利用了全部单词的表示,从不同的粒度提取了特征。我们的这一模型,在三个公开数据集Restaurant、Laptop、Twitter上取得了最佳的结果。

然后,为了在senti-stock这一数据集上取得更好的结果,我们提出了基于attention机制的股票关系模型MSRA(使用attention机制的多股票关系模型, Multi-Stock Relation model using Attention mechanism)。MSRA包括了情感分析模块和股票关系模块。情感分析模块使用多尺度卷积来学习新闻文本的特征。同时,为了考虑股票的相关信息,我们引入了股票系数。股票系数由股票向量通过一个全连接层学习得到,它引入了股票的相关信息。另外,股票之间的相互关系也对股票的变化有着巨大的影响。我们引入了attention机制,利用attention机制学习得到股票之间的相互关系。MSRA在senti-stock数据集上取得了最佳的结果。

% Please add the following required packages to your document preamble:
% \usepackage{multirow}
\begin{table}[ht]
	\centering 
	\caption{基于方面的情感分析主要研究内容}
	\label{tab:content}
	\begin{tabular}{|l|l|l|l|}
	\hline
	任务                                               & 应用场景                  & 数据集         & 模型                                 \\ \hline
	\multirow{3}{*}{ATSA} & \multirow{2}{*}{评论分析} & Restaurant  & \multirow{3}{*}{\textbf{Transformer\&MCNN}} \\ \cline{3-3}
													 &                       & Laptop      &                                    \\ \cline{2-3}
													 & 社交媒体分析                & Twitter     &                                    \\ \hline
	ACSA               & \textbf{股票预测}                  & \textbf{senti-stock} & \textbf{MSRA}                               \\ \hline
	\end{tabular}
\end{table}

表~\ref{tab:content}显示了基于方面的情感分析的主要研究内容。基于方面的情感分析(Aspect-based Sentiment Analysis)旨在提取句子对某一方面或目标的情感极性,又可以细分为对某一目标的情感分析(Aspect-Term Sentiment Analysis, ATSA)和对某一方面的情感分析(Aspect-Category Sentiment Analysis, ACSA)。二者都是一种更细化的情感分析任务,不同的是,对某一目标的情感分析希望提取句子对某一目标的情感极性,这里的目标必须是在句子中出现的一个子串;而对某一方面的情感分析希望提取句子对某一方面的情感极性,这里的方面,并不一定出现在句子当中,而是与句子内容相关的某一方面。比如,“比萨很好吃!”这一句子,对目标“比萨”的情感极性是积极的,对方面“味道”的情感是积极的,“比萨”是出现在句子中的一个子串,而“味道”是与句子内容相关的某一方面,并不一定在句子中出现。在对某一目标的情感分析中,我们提出了使用transformer和多尺度卷积的模型;我们还将对某一方面的情感分析应用于股票预测,构建了senti-stock数据集,提出了MSRA基于attention机制的多股票关系模型。除此之外,我们还调研了大量的文献,总结了新闻驱动的股票预测和基于方面的情感分析之前的研究工作,并复现了多个之前基于方面的情感分析的模型,并将这些模型的实现开源。表格中加粗的部分是我们的主要工作。

  本文的贡献点主要有:
\begin{itemize}
	\item 我们利用路透社的金融新闻和雅虎财经的股票历史数据,创建了新闻驱动的股票预测数据集;
	\item 我们提出了股票向量的概念,并将新闻驱动的股票预测转化为基于方面的情感分析问题;
	\item 提出了基于transformer和多尺度卷积的模型,并在多个开放数据集上取得了最佳的实验效果;
	\item 提出了MSRA模型(使用attention机制的多股票关系模型,Multi-Stock Relation model using Attention mechanism),并在senti-stock数据集上取得了最佳的效果。
\end{itemize}

\section{组织结构}

本文研究基于方面的情感分析及其在新闻驱动的股票预测中的应用\footnote{由于字母符号数量限制,本文中各章节的字母符号可能表示不同的含义,相互不冲突。}。第一章介绍了全文的主要内容;第二章介绍新闻驱动的股票预测和基于方面的情感分析的相关工作;第三章介绍我们所用到的数据集,包括开放数据集Reataurant、Laptop、Twitter,以及我们所创建的新闻驱动的股票预测数据集senti-stock;第四章,我们介绍用于对目标情感分析的模型,使用了transformer结构和多尺度卷积;第五章,我们介绍用于新闻驱动的股票预测模型,提出股票向量的概念,使用了多尺度卷积,利用股票系数来调整句子特征,并利用attention机制学习股票之间的相关关系;第六章介绍了我们的实验设置以及实验结果,我们的模型在公开数据集和股票预测数据集上分别取得了最佳的结果;最后,我们在第七章总结本文的内容。