\chapter{结论}
\label{cha:conclusion}

\section{本文工作总结}

第\ref{cha:introduction}章介绍了研究内容、研究背影以及本文的贡献。本文研究的是新闻驱动的股票预测,即根据新闻事件,来预测股票价格的涨跌。我们为每支股票创建一个股票向量,从而将新闻驱动的股票预测问题转化为了基于方面的情感分析问题(提取句子对某一目标或方面的情感极性)。之前的研究工作使用词典、结构化实体等方式来表示事件,忽略了新闻文本中的大量细节;另外,之前的工作中所用到的数据集大都是不公开的,这给后续工作增加了难度。为解决这些问题,我们利用路透社的新闻和雅虎财经的股票历史数据创建了公开的新闻驱动的股票预测数据集senti-stock,并提出MSRMAM(基于注意力机制的多股票关系模型)。在senti-stock上的实验证明了我们的模型的有效性。

第\ref{cha:relatedwork}章介绍了基于方面的情感分析的相关工作。这些工作主要可以分为注意力(attention)机制、控制门(gate)方法和记忆网络三类。ATAE-LSTM利用attention方法将目标和句子的信息结合起来,提升了LSTM的效果;IAN通过目标对句子的attention和句子对目标的attention进一步提升了效果。BILSTM-ATT-G用目标将句子分为左边、目标和句子三部分,分别用attention方法对三个部分处理,并用gate对三部分的结果加权求和;GCAE利用CNN处理句子,使得模型更加快速,易于并行,另外,该模型使用CNN计算得到的gate,结合了目标和句子的信息。Memnet将问答领域中的memory network引入到了这一领域,将单词向量直接作为memory,每层利用attention方法对memory加权求和;RAM对Memnet进行了改进,用双向LSTM的输出作为memory,引入了位置权重,弥补attention机制无法处理时序信息的缺陷;TNet结合了LSTM和CNN,使用了不同的位置权重和复杂的attention方法。

第\ref{cha:data}章介绍了本文中所用到的数据集,包括开放数据集餐厅(Restaurant)、笔记本电脑(Laptop)、推特(Twitter)和我们所创建的数据集senti-stock。Restaurant和Laptop是SemEval 2014 Task 4中所用到的数据集,由餐厅和笔记本电脑的评论经人工标注而成;Twitter由推特数据经人工标注而成。这三个数据集是在对目标的情感分析中应用最为广泛的数据集。另外,为了解决之前的新闻驱动的股票预测中公开数据集缺乏的问题,我们构建了senti-stock数据集。我们使用scrapy框架,从路透社和雅虎财经上抓取了大量的股票新闻数据和历史数据。我们假设股票的涨跌仅与前一时间段内股票的相关新闻有关,忽略其他因素。根据下一个时间段的涨跌创建标签,新闻文本作为输入句子,股票作为输入的方面(Aspect)。senti-stock数据集可以在我们给出的链接中获得。

第\ref{cha:transformer}章介绍了我们提出的对目标的情感分析的模型。我们使用了转换器transformer结构,它没有使用长短时记忆网络LSTM和卷积神经网络CNN,完全使用自注意力机制self-attention来处理句子信息;转换器transformer由多层结构构成,每层包含自注意力机制self-attention和前向网络。利用转换器transformer构建的语言模型,在大规模语料库上进行预训练后,可以大大提升后续任务的实验效果。我们使用OpenAI提供的GPT模型。我们将句子与目标拼接,从而将对目标的情感分析问题转化为句子对分类问题。一般情况下,仅使用转换器transformer的第一个token的输出作为整个句子的表示,忽视了其他token的有用信息。我们使用深度可分离depthwise separable的多尺度卷积神经网络CNN,结合了全部的信息,并且降低了参数的数量。

第\ref{cha: msra}章介绍了我们的新闻驱动的股票预测模型MSRMAM(使用注意力机制的多股票关系模型)。我们的模型包括情感分析模块和股票关系模块两部分。这里,我们用到了股票向量,即为每支股票创建一个向量化的表示。情感分析模块使用多尺度卷积处理句子的信息,利用全连接网络学习得到股票系数,通过乘积的方式,将股票的信息和情感特征结合起来。股票关系模块通过注意力机制学习股票之间的相关关系。

第\ref{cha:experiment}章介绍了我们的实验设置和实验结果。在对目标的情感分析方面,我们的基于转换器和多尺度卷积的模型,在公开数据集Reataurant、Laptop、Twitter上均取得了不错的效果;我们进行了模型简化实验,去掉预训练后,模型效果降低很多,可见预训练使得转换器学到了大量的知识,对模型效果有巨大的影响;去掉多尺度卷积后,模型效果降低约1个百分点,可见多尺度卷积可以利用全部token的信息,能够提升模型的效果。在新闻驱动的股票预测方面,MSRMAM模型在senti-stock数据集上取得了不错的效果。在模型简化实验中,我们分别删除了模型中的多尺度卷积、股票系数、股票关系模块。从模型简化实验的结果来看,多尺度卷积能够提取不同长度的短语的特征,股票系数能够引入股票的相关关系,都能够提升模型的效果。股票关系在短期预测中作用不明显,在中长期预测中效果更显著。

最后,我们对全文的贡献作出总结。本文研究了基于方面的情感分析问题及其在新闻驱动的股票预测中的应用。基于方面的情感分析旨在提取一个句子对某一方面或目标的情感极性。我们提出了使用转换器和多尺度卷积的模型来解决对某一目标的情感分析问题,并在公开数据集Restaurant、Laptop、Twitter上取得了不错的结果。同时,我们将基于方面的情感分析应用于新闻驱动的股票预测。通过为每个股票创建一个向量化的表示,我们将新闻驱动的股票预测转化为对某一方面的情感分析问题。我们创建了名为senti-stock的数据集,这一数据集由来自于路透社的金融新闻和雅虎财经的股票历史数据构成,共包括三万条数据。我们提出MSRMAM(多股票关系模型),使用带有股票系数的多尺度卷积提取情感信息,利用注意力机制学习股票之间的相关关系。在senti-stock数据集上的实验证明了模型的有效性。

\section{未来研究展望}

股票预测一直是学术界和金融界的研究热点之一。本文,我们关注于新闻驱动的股票预测,即通过分析新闻,来预测股票的涨跌。通过为每支股票创建一个向量的方式,我们将新闻驱动的股票预测转化为了基于方面的情感分析问题,即预测句子(新闻文本)对方面(股票)或目标的情感极性(涨跌)。为了应对之前研究中新闻表示方法忽视大量细节和缺少公开数据集的问题,我们提出了使用注意力机制的多股票关系模型MSRMAM,并创建了公开的新闻驱动股票数据集senti-stock。在senti-stock上的实验证明了模型的有效性。同时,我们也提出基于转换器和多尺度卷积的模型,并在公开的对目标的情感分析数据集上取得了不错的效果。在未来的研究中,我们仍有许多可以改进和提升的地方。

在数据集方面,为构建数据集,我们假设股票的涨跌仅与前一时间段股票的相关新闻有关,而忽视了其他因素。这一假设在现实当中并不一定成立。为应对这一问题,在后续研究中,我们可以考虑引入其他的因素,如股票的历史数据、技术指标、财报等数据和信息。另外,我们的数据集规模虽然已经比之前的基于方面的情感分析数据集提升了一个数量级,但考虑到股票预测的复杂性,我们可以抓取更多的数据,构建更大的数据集,可能会对股票预测效果的提升有很大帮助。

在模型方面,我们除了我们提出的MSRMAM模型之外,也尝试了其他基于LSTM、CNN、转换器的模型,然而实验的效果并不理想。在引入股票历史数据、技术指标、财报等信息后,如何结合这些多模态的信息,将成为新的问题。如何从模型的构建的角度结合多模态信息,可以成为后续研究的重点方向。